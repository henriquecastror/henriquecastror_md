% Options for packages loaded elsewhere
\PassOptionsToPackage{unicode}{hyperref}
\PassOptionsToPackage{hyphens}{url}
%
\documentclass[
]{article}
\usepackage{lmodern}
\usepackage{amssymb,amsmath}
\usepackage{ifxetex,ifluatex}
\ifnum 0\ifxetex 1\fi\ifluatex 1\fi=0 % if pdftex
  \usepackage[T1]{fontenc}
  \usepackage[utf8]{inputenc}
  \usepackage{textcomp} % provide euro and other symbols
\else % if luatex or xetex
  \usepackage{unicode-math}
  \defaultfontfeatures{Scale=MatchLowercase}
  \defaultfontfeatures[\rmfamily]{Ligatures=TeX,Scale=1}
\fi
% Use upquote if available, for straight quotes in verbatim environments
\IfFileExists{upquote.sty}{\usepackage{upquote}}{}
\IfFileExists{microtype.sty}{% use microtype if available
  \usepackage[]{microtype}
  \UseMicrotypeSet[protrusion]{basicmath} % disable protrusion for tt fonts
}{}
\makeatletter
\@ifundefined{KOMAClassName}{% if non-KOMA class
  \IfFileExists{parskip.sty}{%
    \usepackage{parskip}
  }{% else
    \setlength{\parindent}{0pt}
    \setlength{\parskip}{6pt plus 2pt minus 1pt}}
}{% if KOMA class
  \KOMAoptions{parskip=half}}
\makeatother
\usepackage{xcolor}
\IfFileExists{xurl.sty}{\usepackage{xurl}}{} % add URL line breaks if available
\IfFileExists{bookmark.sty}{\usepackage{bookmark}}{\usepackage{hyperref}}
\hypersetup{
  pdftitle={Issues in Empirical Finance Research},
  pdfauthor={Henrique Castro Martins},
  hidelinks,
  pdfcreator={LaTeX via pandoc}}
\urlstyle{same} % disable monospaced font for URLs
\usepackage[margin=1in]{geometry}
\usepackage{graphicx,grffile}
\makeatletter
\def\maxwidth{\ifdim\Gin@nat@width>\linewidth\linewidth\else\Gin@nat@width\fi}
\def\maxheight{\ifdim\Gin@nat@height>\textheight\textheight\else\Gin@nat@height\fi}
\makeatother
% Scale images if necessary, so that they will not overflow the page
% margins by default, and it is still possible to overwrite the defaults
% using explicit options in \includegraphics[width, height, ...]{}
\setkeys{Gin}{width=\maxwidth,height=\maxheight,keepaspectratio}
% Set default figure placement to htbp
\makeatletter
\def\fps@figure{htbp}
\makeatother
\setlength{\emergencystretch}{3em} % prevent overfull lines
\providecommand{\tightlist}{%
  \setlength{\itemsep}{0pt}\setlength{\parskip}{0pt}}
\setcounter{secnumdepth}{-\maxdimen} % remove section numbering

\title{Issues in Empirical Finance Research}
\author{Henrique Castro Martins}
\date{This version - 2020-10-27}

\begin{document}
\maketitle

class: left

\hypertarget{the-challenge}{%
\section{The challenge}\label{the-challenge}}

\[\\[1cm]\]

\begin{itemize}
\tightlist
\item
  I will discuss some issues in using plain OLS models in Corporate
  Finance \& Governance Research
\end{itemize}

--

\begin{itemize}
\tightlist
\item
  I will avoid the word ``endogeneity'' as much as I can
\end{itemize}

--

\begin{itemize}
\tightlist
\item
  I will also avoid the word ``identification'' because identification
  does not guarantee causality and vice-versa (Kahn and Whited 2017)
\end{itemize}

--

\begin{itemize}
\tightlist
\item
  The discussion is based on
  \href{https://www.nowpublishers.com/article/Details/CFR-0036}{Atanasov
  and Black (2016)}
\end{itemize}

\begin{center}\rule{0.5\linewidth}{0.5pt}\end{center}

class: left

\hypertarget{the-challenge-1}{%
\section{The challenge}\label{the-challenge-1}}

\[\\[0.75cm]\]

\begin{itemize}
\item
  Imagine that you want to investigate the effect of Governance on Q

  \begin{itemize}
  \tightlist
  \item
    You may have more covariates explaining Q (omitted from slides)
  \end{itemize}
\end{itemize}

\[\\[0.2cm]\]

\(𝑸_{i} = α + 𝜷_{i} × Gov + Controls + error\)

\[\\[0.2cm]\]

--

All the issues in the next slides will make it not possible to infer
that \textbf{changing Gov will \emph{CAUSE} a change in Q}

That is, cannot infer causality

\begin{center}\rule{0.5\linewidth}{0.5pt}\end{center}

class: left

\hypertarget{reverse-causation}{%
\section{1) Reverse causation}\label{reverse-causation}}

\[\\[0.75cm]\]

\emph{One source of bias is: reverse causation}

\begin{itemize}
\item
  Perhaps it is Q that causes Gov
\item
  OLS based methods do not tell the difference between these two betas:
\end{itemize}

\[\\[0.2cm]\]

\(𝑄_{i} = α + 𝜷_{i} × Gov + Controls + error\)

\(Gov_{i} = α + 𝜷_{i} × Q + Controls + error\)

\[\\[0.2cm]\]

\begin{itemize}
\item
  If one Beta is significant, the other will most likely be significant
  too
\item
  You need a sound theory!
\end{itemize}

\begin{center}\rule{0.5\linewidth}{0.5pt}\end{center}

class: left

\hypertarget{omitted-variable-bias-ovb}{%
\section{2) Omitted variable bias
(OVB)}\label{omitted-variable-bias-ovb}}

\emph{The second source of bias is: OVB}

\begin{itemize}
\item
  Imagine that you do not include an important ``true'' predictor of Q
\item
  Let's say, long is:
  \(𝑸_{i} = 𝜶_{long} + 𝜷_{long}* gov_{i} + δ * omitted + error\)
\item
  But you estimate short:
  \(𝑸_{i} = 𝜶_{short} + 𝜷_{short}* gov_{i} + error\)
\item
  \(𝜷_{short}\) will be:

  \begin{itemize}
  \item
    \(𝜷_{short} = 𝜷_{long}\) + bias
  \item
    \(𝜷_{short} = 𝜷_{long}\) + relationship between omitted (omitted)
    and included (Gov) * effect of omitted in long (δ)

    \begin{itemize}
    \tightlist
    \item
      Where: relationship between omitted (omitted) and included (Gov)
      is: \(Omitted = 𝜶 + ϕ *gov_{i} + u\)
    \end{itemize}
  \end{itemize}
\item
  Thus, OVB is: \(𝜷_{short} – 𝜷_{long} = ϕ * δ\)
\item
  See an example in r
  \href{https://www.youtube.com/watch?v=-Il68vTUI5I}{here}
\end{itemize}

\begin{center}\rule{0.5\linewidth}{0.5pt}\end{center}

\hypertarget{specification-error}{%
\section{3) Specification error}\label{specification-error}}

\[\\[0.75cm]\]

\emph{The third source of bias is: Specification error}

\begin{itemize}
\item
  Even if we could perfectly measure gov and all relevant covariates, we
  would not know for sure the functional form through which each
  influences q

  \begin{itemize}
  \tightlist
  \item
    Functional form: linear? Quadratic? Log-log? Semi-log?
  \end{itemize}
\item
  Misspecification of x's is similar to OVB
\item
  Poor specification leads to bias
\end{itemize}

\begin{center}\rule{0.5\linewidth}{0.5pt}\end{center}

\hypertarget{signaling}{%
\section{4) Signaling}\label{signaling}}

\[\\[0.75cm]\]

\emph{The fourth source of bias is: Signaling}

\begin{itemize}
\item
  Perhaps, some individuals are signaling the existence of an X without
  truly having it:

  \begin{itemize}
  \tightlist
  \item
    For instance: firms signaling they have good governance without
    having it
  \end{itemize}
\item
  This is similar to the OVB because you cannot observe the full story
\end{itemize}

\begin{center}\rule{0.5\linewidth}{0.5pt}\end{center}

\hypertarget{simultaneity}{%
\section{5) Simultaneity}\label{simultaneity}}

\[\\[0.75cm]\]

\emph{The fifth source of bias is: Simultaneity}

\begin{itemize}
\item
  Perhaps gov and some other variable x are determined simultaneously
\item
  Perhaps there is bidirectional causation, with q causing gov and gov
  also causing q
\item
  In both cases, OLS regression will provide a biased estimate of the
  effect
\item
  Also, the sign might be wrong
\end{itemize}

\begin{center}\rule{0.5\linewidth}{0.5pt}\end{center}

\hypertarget{heterogeneous-effects}{%
\section{6) Heterogeneous effects}\label{heterogeneous-effects}}

\[\\[0.75cm]\]

\emph{The sixth source of bias is: Heterogeneous effects}

\begin{itemize}
\item
  Maybe the causal effect of gov on q depends on observed and unobserved
  firm characteristics:

  \begin{itemize}
  \tightlist
  \item
    Let's assume that firms seek to maximize q
  \item
    Different firms have different optimal gov
  \item
    Firms know their optimal gov
  \item
    If we observed all factors that affect q, each firm would be at its
    own optimum and OLS regression would give a non-significant
    coefficient
  \end{itemize}
\item
  In such case, we may find a positive or negative relationship.
\item
  Neither is the true causal relationship
\end{itemize}

\begin{center}\rule{0.5\linewidth}{0.5pt}\end{center}

\hypertarget{construct-validity}{%
\section{7) Construct validity}\label{construct-validity}}

\[\\[0.75cm]\]

\emph{The seventh source of bias is: Construct validity}

\begin{itemize}
\item
  Some constructs (e.g.~Corporate governance) are complex, and sometimes
  have conflicting mechanisms
\item
  We usually don't know for sure what ``good'' governance is, for
  instance
\item
  It is common that we use imperfect proxies
\item
  They may poorly fit the underlying concept
\end{itemize}

\begin{center}\rule{0.5\linewidth}{0.5pt}\end{center}

\hypertarget{measurement-error}{%
\section{8) Measurement error}\label{measurement-error}}

\[\\[0.4cm]\]

\emph{The eighth source of bias is: Measurement error}

\begin{itemize}
\item
  ``Classical'' measurement error for the outcome will inflate standard
  errors but will not lead to biased coefficients.

  \begin{itemize}
  \tightlist
  \item
    \(y^{*} = y + \sigma_{1}\)
  \item
    If you estimante \(y^{*} = f(x)\), you have
    \(y + \sigma_{1} = x + \epsilon\)
  \item
    \(y = x + u\)

    \begin{itemize}
    \tightlist
    \item
      where \(u = \epsilon + \sigma_{1}\)
    \end{itemize}
  \end{itemize}
\item
  ``Classical'' random measurement error in x's will bias coefficient
  estimates toward zero

  \begin{itemize}
  \tightlist
  \item
    \(x^{*} = x + \sigma_{2}\)
  \item
    Imagine that \(x^{*}\) is a bunch of noise
  \item
    It would not explain anything
  \item
    Thus, your results are biased toward zero
  \end{itemize}
\end{itemize}

\begin{center}\rule{0.5\linewidth}{0.5pt}\end{center}

\hypertarget{observation-bias}{%
\section{9) Observation bias}\label{observation-bias}}

\[\\[0.75cm]\]

\emph{The ninth source of bias is: Observation bias}

\begin{itemize}
\item
  This is analogous to the Hawthorne effect, in which observed subjects
  behave differently because they are observed
\item
  Firms which change gov may behave differently because their managers
  or employees think the change in gov matters, when in fact it has no
  direct effect
\end{itemize}

\begin{center}\rule{0.5\linewidth}{0.5pt}\end{center}

\hypertarget{interdependent-effects}{%
\section{10) Interdependent effects}\label{interdependent-effects}}

\[\\[0.75cm]\]

\emph{The tenth source of bias is: Interdependent effects}

\begin{itemize}
\item
  Imagine that a governance reform that will not affect share prices for
  a single firm might be effective if several firms adopt
\item
  Conversely, a reform that improves efficiency for a single firm might
  not improve profitability if adopted widely because the gains will be
  competed away
\item
  ``one swallow doesn't make a summer''
\end{itemize}

\begin{center}\rule{0.5\linewidth}{0.5pt}\end{center}

\hypertarget{selection-bias}{%
\section{11) Selection bias}\label{selection-bias}}

\[\\[0.75cm]\]

\emph{The eleventh source of bias is: Selection bias}

\begin{itemize}
\item
  If you run a regression with two types of companies

  \begin{itemize}
  \tightlist
  \item
    High gov (let's say they are treated)
  \item
    Low gov (let's say they are control)
  \end{itemize}
\item
  Without any matching method, these companies are likely not comparable
\item
  Thus, the estimated beta will contain selection bias
\item
  The bias can be either be positive or negative
\item
  It is similar to OVB
\end{itemize}

\begin{center}\rule{0.5\linewidth}{0.5pt}\end{center}

\hypertarget{self-selection}{%
\section{12) Self-Selection}\label{self-selection}}

\[\\[0.75cm]\]

\emph{The twelfth source of bias is: Self-Selection}

\begin{itemize}
\item
  Self-selection is a type of selection bias
\item
  Usually, firms decide which level of governance they adopt
\item
  There are reasons why firms adopt high governance

  \begin{itemize}
  \tightlist
  \item
    If observable, you need to control for
  \item
    If unobservable, you have a problem
  \end{itemize}
\item
  It is like they ``self-select'' into the treatment

  \begin{itemize}
  \tightlist
  \item
    Units decide whether they receive the treatment of not
  \end{itemize}
\item
  Your coefficients will be biased
\end{itemize}

\begin{center}\rule{0.5\linewidth}{0.5pt}\end{center}

class: right, middle

.left{[} \textbf{I hope you like this class!} {]}

\[\\[2.25cm]\]

\hypertarget{find-me-at}{%
\section{\texorpdfstring{\emph{Find me
at:}}{Find me at:}}\label{find-me-at}}

\href{https://henriquemartins.net/}{henriquemartins.net}

\href{mailto:hcm@iag.puc-rio.br}{\nolinkurl{hcm@iag.puc-rio.br}}

\end{document}
